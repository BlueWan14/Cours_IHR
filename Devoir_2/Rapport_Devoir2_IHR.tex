\documentclass[a4paper,12pt]{article}
\usepackage[french]{babel}
\usepackage[utf8]{inputenc}
\usepackage[T1]{fontenc}
\usepackage{mathptmx}
\usepackage{graphicx}
\usepackage{geometry}
\usepackage{amsmath}
\usepackage{float}
\usepackage{listings}
\usepackage{xcolor}
\usepackage{hyperref}
\usepackage{subcaption}

\setcounter{secnumdepth}{0}
\hypersetup{colorlinks=true, linkcolor=black}
\geometry{margin=1in}

\definecolor{codegreen}{rgb}{0,0.5,0.07}
\definecolor{codeblue}{rgb}{0.05,0,1.0}
\definecolor{codepurple}{rgb}{0.65,0.35,0.96}
\lstdefinestyle{mystyle}{
    commentstyle=\color{codegreen},
    keywordstyle=\color{codeblue},
    stringstyle=\color{codepurple},
    basicstyle=\fontfamily{pcr}\normalsize,
    breakatwhitespace=true,
    breaklines=true,
    frame=lines,
}
\lstset{style=mystyle}

\captionsetup{font={
    footnotesize,
    color=gray
}}

\begin{document}
    \selectlanguage{french}
    
    \begin{titlepage}
        \begin{center}
            % Logo de l'école en en-tête
            \includegraphics[width=5cm]{./img/uqac.png}\\[1cm]
            
            % Titre principal
            \vspace*{1cm} % Ajuster pour centrer verticalement
            {\Huge \textbf{Interaction Humain-Robot : Devoir 2}\\[0.5cm]}
    
            % Image de couverture (centrée en dessous du logo)
            \vspace*{1cm}
            \includegraphics[width=0.7\textwidth]{./img/image_IHR.png}\\[1cm]
    
            \vspace{4.5cm}
            \begin{tabular*}{1\linewidth}{@{\extracolsep{\fill}}l c r}
                {\small Constance ALOYAU} &  & {\small ALOC25530200} \\
                {\small Erwan MAWART} & {\normalsize 16 novembre 2024} & {\small MAWE14050200} \\
                {\small Benjamin PELLIEUX} &  & {\small PELB28120100} \\
            \end{tabular*}
            
            \vfill
        \end{center}
    \end{titlepage}
    
    \tableofcontents
    \newpage
    
    \section{Introduction}
    Les vibrations dans un mécanisme robotique manipulé par un opérateur humain peuvent significativement compromettre la précision et la qualité de la tâche effectuée. Lorsque l’opérateur applique une force via une poignée équipée d’un capteur, des vibrations non désirées peuvent survenir, surtout lorsque l'opérateur ajuste la rigidité de son bras. Ces vibrations perturbent non seulement la performance mais réduisent aussi le confort de l’utilisateur.

    Dans le cadre de ce projet, nous cherchons à modéliser et analyser l'impact perceptuel de ces vibrations en interaction humain-robot. Notre approche implique le développement d'un observateur de vibrations capable de détecter et segmenter les parties du signal problématiques, permettant ainsi un ajustement dynamique d’un contrôleur proportionnel en fonction de l'indice de vibration mesuré. L’objectif est de minimiser ces vibrations, tout en tenant compte de l’expérience ressentie par l’opérateur.

    Ce rapport aborde plusieurs aspects de la conception et de la simulation du modèle du mécanisme, incluant la formulation des équations d’état du système, l’implémentation d’un critère de stabilité basé sur le critère de Routh-Hurwitz, ainsi que la configuration de la boucle de contrôle pour limiter les vibrations. Nous explorons également la détection autonome des caractéristiques vibratoires à travers une simulation pour tester l'efficacité de notre modèle dans des scénarios variés.
    
    \newpage
    \section{Énoncé 1 : Conception du modèle du mécanisme}
    \subsection{Question 1.1}
    \begin{figure}[H]
        \centering
        \includegraphics[width=14cm]{./img/modele_humain-meca_robotique.png}
        \caption{
            Modèle du mécanisme robotique et de l'humain 
            \label{fig:ModelMecaRobotHumain}
        }
    \end{figure}
    
    La \autoref{fig:ModelMecaRobotHumain} présente le mécanisme robotique que nous allons utiliser, où :
    \begin{itemize}
        \item[$\bullet$] $M_R$ : masse de la charge \textit{[kg]}.
        \item[$\bullet$] $m_R$ : masse du système \textit{[kg]}.
        \item[$\bullet$] $K_B$ : constante de raideur des courroies \textit{[N/m]}.
        \item[$\bullet$] $C_B$ : coefficient d'amortissement des courroies \textit{[$N\cdot s/m$]}.
        \item[$\bullet$] $K_H$ : coefficient de raideur de l'humain\textit{[N/m]}.
        \item[$\bullet$] $C_H$ : coefficient d'amortissement de l'humain \textit{[$N\cdot s/m$]}.
        \item[$\bullet$] $C_R$ : coefficient de frottement \textit{[$N\cdot s/m$]}.
        \item[$\bullet$] $F$ : Force commandée par le système \textit{[N]}.
        \item[$\bullet$] $f_H$ : Force appliquée par l'humain \textit{[N]}.
        \item[$\bullet$] $x_1$ : position des moteurs \textit{[m]}.
        \item[$\bullet$] $x_2$ : position de la charge \textit{[m]}.
        \item[$\bullet$] $v_1$ : vitesse des moteurs \textit{[m/s]}, la dérivée de la position ($v_1 = \dot{x_1}$).
        \item[$\bullet$] $v_2$ : vitesse de la charge \textit{[m/s]}, la dérivée de la position ($v_2 = \dot{x_2}$).
    \end{itemize}
    
    Ainsi, on déduit les variables d'états ci-dessous.
    \[
        X =
        \begin{bmatrix}
            x_1 \\
            x_2 \\
            v_1 \\
            v_2
        \end{bmatrix}
        =
        \begin{bmatrix}
            x_1 \\
            x_2 \\
            \dot{x_1} \\
            \dot{x_2}
        \end{bmatrix}
    \]
    
    En effectuant la somme des forces appliquées sur chaque masse, on obtient le système suivant.
    \[
        \begin{cases}
            \sum F_{ext/m_R} = m_R\ddot{x_1} \\
            \sum F_{ext/M_R} = M_R\ddot{x_2} \\
        \end{cases}
        \Leftrightarrow
        \begin{cases}
            m_R\ddot{x_1} = F - K_B x_1 + K_B x_2 - C_B \dot{x_1} + C_B\dot{x_2} \\
            M_R\ddot{x_2} = K_B x_1 - K_B x_2 + C_B \dot{x_1} - C_B\dot{x_2} - C_R \dot{x_2} \\
        \end{cases}
    \]
    
    \begin{equation}
        \\ \Leftrightarrow
        \begin{cases}
            \ddot{x_1} = \frac{F - K_B x_1 + K_B x_2 - C_B \dot{x_1} + C_B \dot{x_2}}{m_R} \\
            \ddot{x_2} = \frac{K_B x_1 - K_B x_2 + C_B \dot{x_1} - C_B \dot{x_2} - C_R \dot{x_2}}{M_R}
        \end{cases}
        \label{eq:SommeDesForces}
    \end{equation}
    
    Grâce à celle-ci, on construit les deux équations de fonctionnement du système :
    
    \begin{equation}
        \dot{X}=AX+BU
        \Leftrightarrow
        \begin{bmatrix}
            \dot{x_1}\\
            \dot{x_2}\\
            \ddot{x_1}\\
            \ddot{x_2}
        \end{bmatrix}
        =
        \begin{bmatrix}
            0 & 0 & 1 & 0\\
            0 & 0 & 0 & 1\\
            \frac{-K_B}{m_R} & \frac{K_B}{m_R} & \frac{-C_B}{m_R} & \frac{C_B}{m_R}\\
            \frac{K_B}{M_R} & \frac{-K_B}{M_R} & \frac{C_B}{M_R} & \frac{-(C_B+C_R)}{M_R}\\
        \end{bmatrix}
        \begin{bmatrix}
            x_1\\
            x_2\\
            \dot{x_1}\\
            \dot{x_2}
        \end{bmatrix}
        +
        \begin{bmatrix}
            0\\
            0\\
            \frac{1}{m_R}\\
            0
        \end{bmatrix}
        \cdot F
        \label{eq:VarEtatXDot}
    \end{equation}
    
    \begin{equation}
        Y=CX+DU
        \Leftrightarrow
        Y=CX
        \Leftrightarrow
        \begin{bmatrix}
            x_2\\
            \dot{x_1}
        \end{bmatrix}
        =
        \begin{bmatrix}
            0 & 1 & 0 & 0\\
            0 & 0 & 1 & 0
        \end{bmatrix}
        \begin{bmatrix}
            x_1\\
            x_2\\
            \dot{x_1}\\
            \dot{x_2}
        \end{bmatrix}\\
        \label{eq:VarEtatY}
    \end{equation}
    
    L'\autoref{eq:VarEtatY} présente la sortie du système. Dans notre cas, on utilise $x_2$ (la position de la charge) et $v_1$ (la vitesse des moteurs) afin d'asservir notre système. Où $x_2$ permet à l'humain de corriger la trajectoire de la charge et $v_1$ d'asservir les moteurs.
    
    
    \subsection{Question 1.2}
    Lors d'une étude précédente, nous nous penchions sur la détectection des vibrations générées par le système lorsque l'humain se régédifie. Pour cela, nous analysions la vitesse des moteurs pour en extraire des caractéristiques temporelles et/ou fréquentielles qui nous permettent de déterminer la présence des vibrations. \\
    
    \begin{figure}[H]
        \centering
        \includegraphics[width=16cm]{./img/SchemaBlocAvecObs.png}
        \caption{Commande des mécanisme robotique et humain avec observateur\label{fig:SchemaBlocAvecObs}}
    \end{figure}
    
    Afin de rendre le système autonome, il est donc nécessaire de rendre automatique cette recherche de caractéristiques. C'est pourquoi, comme le montre la \autoref{fig:SchemaBlocAvecObs}, on ajoute un observateur qui va influer sur la valeur du gain $K_p$, et donc, limiter les vibrations générées par le système.
    
    Dans Simulink, on représentera ce bloc avec l'architecture et la fonction contenues dans l'\hyperref[Annexe:modelObs]{annexe 7}.
    
    
    \subsection{Question 1.3}
    Le critère de Routh-Hurwitz est une méthode mathématique permettant déterminer rapidement la stabilité d'un système. Elle consiste à générer un tableau à partir des coefficients du dénominateur de la fonction de transfert tel que présenté au \autoref{tab:table_theo_critRH}. Une fois construit, on observe sa première colonne. Le nombre de changement de signe détermine la quantité de racines étant dans la partie astable du plan-s (c'est-à-dire dans la partie réel positive).
    
    \begin{table}[ht]
        \centering
        \[
        \begin{array}{|c|c c c c|}
            \hline
            s^n & a_n & a_{n-2} & a_{n-4} & \dots \\
            \hline
            s^{n-1} & a_{n-1} & a_{n-3} & a_{n-5} & \dots \\
            \hline
            s^{n-2} & b_{n-1} & b_{n-3} & b_{n-5} & \dots \\
            \hline
            s^{n-3} & c_{n-1} & c_{n-3} & c_{n-5} & \dots \\
            \hline
            \vdots & \vdots & \vdots & \vdots & \vdots \\
            \hline
            s^0 & h_{n-1} & h_{n-3} & h_{n-5} & \dots \\
            \hline
        \end{array}
        \]
        \caption{Tableau théorique du critère de Routh-Hurwitz}
        \label{tab:table_theo_critRH}
    \end{table}
    
    Dans le \autoref{tab:table_theo_critRH}, on retrouve :
    \begin{itemize}
        \item[$\bullet$] $a_{n-i}$, les coefficients du dénominateur.
        \item[$\bullet$] $b_{n-i}$, donnés par :
            $b_{n-i} = \frac{-1}{a_{n-1}}
            \begin{vmatrix}
                a_n & a_{n-i-1} \\
                a_{n-1} & a_{n-i-2}
            \end{vmatrix}
            = \frac{a_{n-1} a_{n-i-1} - a_n a_{n-i-2}}{a_{n-1}}$
        \item[$\bullet$] $c_{n-i}$, donnés par :
            $c_{n-i} = \frac{-1}{b_{n-1}}
            \begin{vmatrix}
                a_n & a_{n-i-2} \\
                b_{n-1} & b_{n-i-2}
            \end{vmatrix}
            = \frac{b_{n-1} a_{n-i-2} - a_n b_{n-i-2}}{b_{n-1}}$
        \item[$\bullet$] etc.
    \end{itemize} 
    
    On utilise cette méthode afin de déterminer la plage de valeurs des gain $K_p$. Afin de construire cette table plus simplement, on utilise cette fonction dans Matlab : \\
    \begin{lstlisting}[label={code:fctCritRH}, caption={Fonction Matlab génération du critère de Routh-Hurwitz}, language=Matlab]
    %%%%%%%%%%%%%%%%%%%%%%%%%%%%%%%%%%%%%%%%%%%%%%%%%%%%%%%
    % Parametres :                                        %
    %   - TF : la fonction de transfert a analyser        %
    %   - Oldvars : les variables a remplacer             %
    %   - Newvars : les valeurs a implementer             %
    %   - s : la variable de Laplace                      %
    % Sortie :                                            %
    %   - S : le tableau construit                        %
    %%%%%%%%%%%%%%%%%%%%%%%%%%%%%%%%%%%%%%%%%%%%%%%%%%%%%%%
    function S = calcTabRH(TF, Oldvars,  Newvars, s)
        % Extraction des coefficients du denominateur =====
        [~, Den] = numden(TF);
        [an, terms] = coeffs(Den, s);
        l_terms = length(terms);
        l = round(l_terms/2);

        % Creation du tableau =============================
        S = sym(zeros(l_terms, l));
        % Ajout des coefficients du denominateur
        for n=1:l
            i = n*2;
            S(l_terms, n) = an(i-1);
            if i < l_terms
                S(l_terms-1, n) = an(i);
            end
        end
        % Calcul des autres elements du tableau
        for k=l_terms-2:-1:1
            for n=1:l-1
                S(k, n) = (S(k+1, 1)*S(k+2, n+1)-S(k+2, 1)*S(k+1, n+1))/S(k+1, 1);
            end
        end

        % Remplacement des valeurs connues ================
        S = simplify(subs(S, Oldvars, Newvars));

        % Affichage du tableau ============================
        disp("Tableau du critere de Routh-Hurwitz :")
        disp(S)
    end
    \end{lstlisting}

    Les gains $K_p$ et $K_H$ étant dépendants l'un de l'autre, leur plage de valeurs varie en fonction de l'autre. Par conséquent, il nous est obligatoire de fixer l'un des deux. Ici, fixons $K_H = 50 N/m$, s'assurant ainsi la stabilité du système.
    
    En utilisant les valeurs données en \hyperref[Annexe:ValList]{annexe 1} dans le tableau du critère de Routh-Hurwitz, on obtient donc cinq coefficients contenant $K_p$. Parmis ceux-ci, le critère $S^2$ devient négatif à $K_p = 90$, comme présenté à la \autoref{fig:CoefS2}. Ainsi, on déduit que $K_p \in \left< 0, 90 \right>$.
    \begin{figure}[H]
        \centering
        \includegraphics[width=16cm]{./img/plage_coefKp.png}
        \caption{Critère de Routh-Hurwitz $S^2$ en fonction de $K_p$\label{fig:CoefS2}}
    \end{figure}

    Afin de confirmer cette valeur, nous pouvons utiliser le lieu des racines. Cette méthode consiste à faire varier un gain en entrée du système afin de déterminer le moment auquel ce-dernier devient astable.
    
    Dans notre cas, le gain variable est $K_H$ et on fixe $K_p = 90$, condition limite déterminée à la \autoref{fig:CoefS2}. En traçant le lieu des racines, comme montré à la \autoref{fig:LieuRacineKh}, on remarque que le système est stable jusqu'à $K_H \approx 50$ ($K_H = 46.9532$ au point de mesure noté sur la \autoref{fig:LieuRacineKh}).
    \begin{figure}[H]
        \centering
        \includegraphics[width=13cm]{./img/planRacine_Kh.png}
        \caption{Lieu des racines du système\label{fig:LieuRacineKh}}
    \end{figure}



    \newpage
    \section{Énoncé 2 : Simulation}
    \subsection{Question 2.1}
    
    \subsection{Question 2.2}
    
    \subsection{Question 2.3}
    
    
    
    \newpage
    \section{Annexe}
    \subsection{Annexe 1 : Liste des valeurs fixes} \label{Annexe:ValList}
    \begin{itemize}
        \item[] \makebox[5cm][l]{\makebox[.6cm][l]{$M_R$} = 500 $kg$} \textit{(masse de la charge)}
        \item[] \makebox[5cm][l]{\makebox[.6cm][l]{$m_R$} = 50 $kg$} \textit{(masse du système)}
        \item[] \makebox[5cm][l]{\makebox[.6cm][l]{$m_v$} = 20 $kg$} \textit{(masse ressentie)}
        \item[] \makebox[5cm][l]{\makebox[.6cm][l]{$K_B$} = 40000 $N/m$} \textit{(constante de raideur des courroies)}
        \item[] \makebox[5cm][l]{\makebox[.6cm][l]{$C_B$} = 40 $N\cdot s/m$} \textit{(coefficient d'amortissement des courroies)}
        \item[] \makebox[5cm][l]{\makebox[.6cm][l]{$C_H$} = 23.45 $N\cdot s/m$} \textit{(coefficient d'amortissement de l'humain)}
        \item[] \makebox[5cm][l]{\makebox[.6cm][l]{$C_R$} = 100 $N\cdot s/m$} \textit{(coefficient de frottement)}
        \item[] \makebox[5cm][l]{\makebox[.6cm][l]{$c_v$} = 20 $N\cdot s/m$} \textit{(coefficient d'amortissement vituel)}
        \item[] \makebox[5cm][l]{\makebox[.6cm][l]{$T$} = 0.1 $s$} \textit{(temps d'échantillonnage)}
    \end{itemize}


    \subsection{Annexe 2 : Fichier de calculs des plages de $K_p$ et $K_H$} \label{Annexe:calcRHFile}
    \begin{lstlisting}[caption={Fonction simulink de calcul du compensateur}, language=Matlab]
    % Initialisation ======================================
    close all
    system_config
    clear Kp Kh
    FontName = 'Times';

    % Calcul des boucles ==================================
    [Boucle_ouverte, Boucle] = Calc_Sys();

    % Creation de la table de Routh-Hurwitz ===============
    syms Kp Kh s
    S_boucle = calcTabRH(Boucle, [sym('MR') sym('mR') sym('Kb') sym('Cb') sym('CR') sym('T') sym('mv') sym('cv')], [MR mR Kb Cb CR T mv cv], s);

    % Recherche de la plage de valeurs Kp =================
    Kp = 0:1500;
    Kh = 50;
    figure;
    hold on
    for i = 1:length(S_boucle(:,1))
        if ~isempty(find(symvar(S_boucle(i,1)) == sym('Kp'), 1))
            S_boucle(i,1) = subs(S_boucle(i,1), sym('Kh'), Kh);
            S_calc = eval(subs(S_boucle(i,1), sym('Kp'), Kp));
    
            subplot(3, 2, i)
            plot(Kp, S_calc, 'LineWidth', 2)
            yline(0, '--')
            xlabel('Kp')
            ylabel("Critere de Routh-Hurwitz S^" + i)
            fontname(FontName)
        end
    end
    hold off
    
    % Recherche de la plage de valeurs Kh =================
    TF = subs( ...
        Boucle_ouverte, ...
        [sym('MR') sym('mR') sym('Kb') sym('Cb') sym('CR') sym('T') sym('mv') sym('cv') sym('Kp')], ...
        [MR mR Kb Cb CR T mv cv 90] ...
    );
    clear s
    TFFun = matlabFunction(TF);
    TFFun = str2func(regexprep(func2str(TFFun), '\.([/^\\*])', '$1'));
    figure;
    sys = tf(TFFun(tf('s')));
    rlocus(sys)
    title("")
    xlabel("Imaginaires")
    ylabel("Reels")
    axis([-0.5, 0.5, -2, 2])
    fontname(FontName)
    [K, poles] = rlocfind(sys);
    

    %%%%%%%%%%%%%%%%%%%%%%%%%%%%%%%%%%%%%%%%%%%%%%%%%%%%%%%
    % Parametres :                                        %
    %   (Aucun)                                           %
    % Sortie :                                            %
    %   - Boucle_NoKh : boucle ouverte                    %
    %   - Boucle1 : boucle fermee                         %
    %%%%%%%%%%%%%%%%%%%%%%%%%%%%%%%%%%%%%%%%%%%%%%%%%%%%%%%
    function [Boucle_NoKh, Boucle1] = Calc_Sys()
        syms MR mR Kb Cb CR Kh Kp T mv cv s

        % Calcul de la boucle de vitesse ==================
        BoucleVitesse0 = Kp*(MR*s^2+Kb+Cb*s+CR*s)*s*1 / (s * (mR*s^3*MR + mR*s*Kb + mR*s^2*Cb + mR*s^2*CR + Kb*MR*s + Kb*CR + Cb*s^2*MR + Cb*s*CR) * (T*s+1));
        BoucleVitesse1 = collect(simplify(BoucleVitesse0/(1+BoucleVitesse0)), s);
        
        Humain = Kh;
        Admittance = 1/(mv*s+cv);
        % Calcul de la boucle complete ====================
        Boucle0 = Humain * Admittance * BoucleVitesse1 * ((MR*s^2+Kb+Cb*s+CR*s)*s)^(-1) * (Kb+Cb*s);
        Boucle_NoKh = Admittance * BoucleVitesse1 * ((MR*s^2+Kb+Cb*s+CR*s)*s)^(-1) * (Kb+Cb*s);
        Boucle1 = collect(simplify(Boucle0*(1+Boucle0)^(-1)), s);
    end
    
    %%%%%%%%%%%%%%%%%%%%%%%%%%%%%%%%%%%%%%%%%%%%%%%%%%%%%%%
    % Parametres :                                        %
    %   - TF : la fonction de transfert a analyser        %
    %   - Oldvars : les variables a remplacer             %
    %   - Newvars : les valeurs a implementer             %
    %   - s : la variable de Laplace                      %
    % Sortie :                                            %
    %   - S : le tableau construit                        %
    %%%%%%%%%%%%%%%%%%%%%%%%%%%%%%%%%%%%%%%%%%%%%%%%%%%%%%%
    function S = calcTabRH(TF, Oldvars,  Newvars, s)
        % Extraction des coefficients du denominateur =====
        [~, Den] = numden(TF);
        [an, terms] = coeffs(Den, s);
        l_terms = length(terms);
        l = round(l_terms/2);

        % Creation du tableau =============================
        S = sym(zeros(l_terms, l));
        % Ajout des coefficients du denominateur
        for n=1:l
            i = n*2;
            S(l_terms, n) = an(i-1);
            if i < l_terms
                S(l_terms-1, n) = an(i);
            end
        end
        % Calcul des autres elements du tableau
        for k=l_terms-2:-1:1
            for n=1:l-1
                S(k, n) = (S(k+1, 1)*S(k+2, n+1)-S(k+2, 1)*S(k+1, n+1))/S(k+1, 1);
            end
        end

        % Remplacement des valeurs connues ================
        S = simplify(subs(S, Oldvars, Newvars));

        % Affichage du tableau ============================
        disp("Tableau du critere de Routh-Hurwitz :")
        disp(S)
    end
    \end{lstlisting}


    \subsection{Annexe 3 : Fichier de configuration de la simulation} \label{Annexe:configSimu}
    \begin{lstlisting}[caption={Fonction simulink de calcul du compensateur}, language=Matlab]
    clear;
    clc;

    %Temps de simulation
    sim_time = 200;     % s
    % Taille de la memoire tampon de l'observateur
    buffer_size = 128;

    % Bloc "Modele humain" ================================
    Kh = 550;           % N/m
    Ch = 23.45;         % N*s/m
    
    % Admittance ==========================================
    cv = 20;            % N*s/m
    mv = 20;            % kg
    
    % Imperfections =======================================
    T = .1;             % s

    % Mecanisme ===========================================
    mR = 50;            % kg
    MR = 500;           % kg
    CR = 100;           % N*s/m
    Cb = 40;            % N*s/m
    Kb = 40000;         % N/m
    [A, B, C, D] = calcIAD(Kb, Cb, CR, mR, MR);

    % Observateur =========================================
    % Ecart-type limite avant vibrations
    ec_max = 1e-3;

    
    %%%%%%%%%%%%%%%%%%%%%%%%%%%%%%%%%%%%%%%%%%%%%%%%%%%%%%%
    % Parametres :                                        %
    %   - K : ici Kb                                      %
    %   - C1 : ici Cb                                     %
    %   - C2 : ici CR                                     %
    %   - m : ici mR                                      %
    %   - M : ici MR                                      %
    % Sortie :                                            %
    %   - A : la matrice d'etat                           %
    %   - B : la matrice de commande                      %
    %   - C : la matrice d'observation                    %
    %   - D : la matrice d'action directe                 %
    %%%%%%%%%%%%%%%%%%%%%%%%%%%%%%%%%%%%%%%%%%%%%%%%%%%%%%%
    function [A, B, C, D] = calcIAD(K, C1, C2, m, M)
        mK = K / m;
        mC = C1 / m;
        MK = K / M;
    
        A = [
             0      0       1       0;
             0      0       0       1;
            -mK     mK      -mC     mC;
             MK     -MK     C1/M    -(C1+C2)/M
            ];
    
        B = [
             0;
             0;
             1/m;
             0
            ];
    
        C = [
             0      1       0       0;
             0      0       1       0;
            ];
    
        D = zeros(2, 1);
    end
    \end{lstlisting}


    \subsection{Annexe 4 : Modèle complet} \label{Annexe:modelComplet}
    \begin{center}
        \includegraphics[width=16cm]{./img/model_complet.png}
    \end{center}
    Où le bloc "step" \textbf{X} est paramétré avec : 
    \begin{itemize}
        \item[$\bullet$] \makebox[2.3cm][l]{"step time"} = 0.5
        \item[$\bullet$] \makebox[2.3cm][l]{"initial value"} = 0
        \item[$\bullet$] \makebox[2.3cm][l]{"final value"} = 0.2
    \end{itemize}


    \subsection{Annexe 5 : Modèle humain} \label{Annexe:modelHumain}
    \begin{center}
        \includegraphics[width=10cm]{./img/model_humain.png}
    \end{center}
    Où les blocs "step" (à gauche) sont paramétrés avec :
    
    \begin{tabular*}{1\linewidth}{@{\extracolsep{\fill}}l l l}
        \multicolumn{1}{c}{Step haut} & \multicolumn{1}{c}{Step milieu} &\multicolumn{1}{c}{Step bas} \\
        \hline
    
        $\bullet$ \makebox[2.3cm][l]{"step time"} = 0 &
        $\bullet$ \makebox[2.3cm][l]{"step time"} = 66 &
        $\bullet$ \makebox[2.3cm][l]{"step time"} = 132 \\
        
        $\bullet$ \makebox[2.3cm][l]{"initial value"} = 0 & 
        $\bullet$ \makebox[2.3cm][l]{"initial value"} = 0 & 
        $\bullet$ \makebox[2.3cm][l]{"initial value"} = 0 \\
        
        $\bullet$ \makebox[2.3cm][l]{"final value"} = 27.5 &
        $\bullet$ \makebox[2.3cm][l]{"final value"} = 550 &
        $\bullet$ \makebox[2.3cm][l]{"final value"} = 577.5 \\
    \end{tabular*}
    

    \subsection{Annexe 6 : Modèle IAD} \label{Annexe:modelIAD}
    \begin{center}
        \includegraphics[width=16cm]{./img/model_IAD.png}
    \end{center}
    

    \subsection{Annexe 7 : Modèle observateur} \label{Annexe:modelObs}
    \begin{center}
        \includegraphics[width=12cm]{./img/model_observateur.png}
    \end{center}
    Où :
    \begin{itemize}
        \item[] Le bloc "rate transition" est paramétré avec : 
            \subitem $\bullet$ \makebox[4.2cm][l]{"initial conditions"} = 0
            \subitem $\bullet$ \makebox[4.2cm][l]{"ouput port sample time"} = 0.01
        \item[] Le bloc "buffer" est paramétré avec : 
            \subitem$\bullet$ \makebox[5.5cm][l]{"ouput buffer size (per channel)"} = 128
            \subitem$\bullet$ \makebox[5.5cm][l]{"buffer overlap"} = 64
            \subitem$\bullet$ \makebox[5.5cm][l]{"initial conditions"} = 0
        \item[] La variable $ec_{max} = 10^{-3}$
        \item[] Et le bloc "MATLAB Function" contient :
    \end{itemize}
    \begin{lstlisting}[caption={Fonction simulink de calcul du compensateur}, language=Matlab]
    %%%%%%%%%%%%%%%%%%%%%%%%%%%%%%%%%%%%%%%%%%%%%%%%%%%%%%%
    % Parametres :                                        %
    %   - u : segment de signal                           %
    %   - ec_max : ecart-type limite avant vibration      %
    % Sortie :                                            %
    %   - y : le coefficient de compensation              %
    %   - ec : l'ecart-type du segment                    %
    %%%%%%%%%%%%%%%%%%%%%%%%%%%%%%%%%%%%%%%%%%%%%%%%%%%%%%%
    function [y, ec] = fcn(u, ec_max)
        % Calcul de l'ecart-type ==========================
        ec = std(u);

        % Calcul de la compensation =======================
        y = 1 - ec * (1/ec_max);
    end
    \end{lstlisting}
    
    
    \subsection{Lien vers le Dépôt GitHub}
    \url{https://github.com/BlueWan14/Cours_IHR/tree/main/Devoir_2}
    
\end{document}
